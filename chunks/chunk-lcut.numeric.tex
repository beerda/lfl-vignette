\begin{Schunk}
% --begin: "lcut.numeric"
\begin{Sinput}
R> age <- c(runif(n = 5, min = 18, max = 65))
R> print(age)
\end{Sinput}
\begin{Soutput}
[1] 54.88504 31.71075 40.91489 52.24082 50.63086
\end{Soutput}
\begin{Sinput}
R> lcut(age, context = ctx3(low = 0, high = 100))
\end{Sinput}
\begin{Soutput}
     ex.sm.age si.sm.age ve.sm.age sm.age ml.sm.age ro.sm.age qr.sm.age
[1,]         0         0         0      0         0         0 0.0000000
[2,]         0         0         0      0         0         0 0.1581754
[3,]         0         0         0      0         0         0 0.0000000
[4,]         0         0         0      0         0         0 0.0000000
[5,]         0         0         0      0         0         0 0.0000000
     vr.sm.age  ty.me.age    me.age ml.me.age ro.me.age qr.me.age vr.me.age
[1,]  0.000000 0.05919657 1.0000000 1.0000000         1         1         1
[2,]  0.957025 0.00000000 0.3679403 0.7717577         1         1         1
[3,]  0.207952 0.00000000 0.9844039 1.0000000         1         1         1
[4,]  0.000000 0.66524793 1.0000000 1.0000000         1         1         1
[5,]  0.000000 0.97346813 1.0000000 1.0000000         1         1         1
     ex.bi.age si.bi.age ve.bi.age bi.age ml.bi.age ro.bi.age qr.bi.age
[1,]         0         0         0      0         0         0         0
[2,]         0         0         0      0         0         0         0
[3,]         0         0         0      0         0         0         0
[4,]         0         0         0      0         0         0         0
[5,]         0         0         0      0         0         0         0
     vr.bi.age
[1,]         0
[2,]         0
[3,]         0
[4,]         0
[5,]         0
\end{Soutput}
%
% --end: "lcut.numeric"
\end{Schunk}
